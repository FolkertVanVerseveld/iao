%%%%%%%%%%%%%%%%%%%%%%%%%%%%%%
% LATEX-TEMPLATE PRESENTATIE
%-------------------------------------------------------------------------------
% Voor informatie over presenterenn, zie
% http://practicumav.nl/presenteren/presenteren.html
% Voor readme en meest recente versie van het template, zie
% https://gitlab-fnwi.uva.nl/informatica/LaTeX-template.git
% Gebaseerd op een template van: http://www.LaTeXTemplates.com
% Licentie: CC BY-NC-SA 3.0
% (http://creativecommons.org/licenses/by-nc-sa/3.0/)
%%%%%%%%%%%%%%%%%%%%%%%%%%%%%%

%-------------------------------------------------------------------------------
%	PACKAGES EN CONFIGURATIE
%-------------------------------------------------------------------------------

%\documentclass[aspectratio=43]{uva-inf-presentation}
\documentclass[aspectratio=169]{beamer}
%\documentclass{beamer}
\usepackage[dutch]{babel}
\usepackage{csquotes}

\title{Natuurrampensimulator op de Commodore 64}
\author{Folkert van Verseveld}

\begin{document}
\maketitle

%-------------------------------------------------------------------------------
%	PRESENTATIE SLIDES
%-------------------------------------------------------------------------------

%------------------------------------------------

\begin{frame}
\frametitle{Introductie}

\begin{itemize}
\item Folkert van Verseveld
\item Tweedejaars Bachelor informaticastudent, studentassistent, computermuseum
\item Verzamel oude computers (apple ][, c64, c128, amiga 500, macintosh, \dots)
\item C64 workshop bij VIA
\end{itemize}

\end{frame}

\begin{frame}
\frametitle{Context}

Academisch onderzoek op unieke wijze presenteren
\begin{itemize}
\item Kijkdoos, toneelstuk, poster
\item Video game: text adventure, graphical game
\item Groepen van 2 tot 6 studenten
\end{itemize}

\end{frame}

\begin{frame}
\frametitle{De Enige Echte Broodtrommel}

\begin{figure}
%\includegraphics[width=0.55\linewidth]{c64.jpg}
\includegraphics[width=0.45\linewidth]{c64.jpg}
\end{figure}

\begin{figure}
%\includegraphics[width=0.55\linewidth]{c64c.jpg}
\includegraphics[width=0.45\linewidth]{c64c.jpg}
\end{figure}

\end{frame}

\begin{frame}
\frametitle{De Enige Echte Broodtrommel}

\begin{figure}
%\includegraphics[width=0.85\linewidth]{breadbin.jpg}
\includegraphics[width=0.75\linewidth]{breadbin.jpg}
\end{figure}

\end{frame}

\begin{frame}
\frametitle{Vintage is the new old}

\begin{figure}
%\includegraphics[width=0.55\linewidth]{retro.png}
\includegraphics[width=0.50\linewidth]{retro.png}
\end{figure}

\end{frame}

\begin{frame}
\frametitle{Vintage is the new old}

\begin{figure}
%\includegraphics[width=0.90\linewidth]{x.jpg}
\includegraphics[width=0.80\linewidth]{x.jpg}
\end{figure}

\end{frame}

\begin{frame}
\frametitle{64KB is weinig}

\begin{figure}
%\includegraphics[width=0.90\linewidth]{mmap.png}
\includegraphics[width=0.70\linewidth]{mmap.png}
\end{figure}

\end{frame}

\begin{frame}
\frametitle{Realisatie}

\begin{itemize}
\item Groep van 6 informaticastudenten
\item 4 coders, 1 schrijver, 2 grafisch ontwerpers, 2 testers
\item 5-6 weken
\item $\sim$9500 regels assembly code
\item Demo!
\end{itemize}

\end{frame}

\end{document}
